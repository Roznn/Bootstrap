%% lots of mistakes -
\frame
{
\frametitle{Bias of $\hat{\theta}$}

\begin{itemize}
\item A large bias is usually an undesirable aspect of an estimator's performance. \alert{Unbiased estimates} (such $\mathbb{E}_f(\hat{\theta})=\theta$) are interesting in practice as they promote a nice feeling of scientific objectivity in the estimation process. 
%\item Plug- in estimates $\hat{\theta}=t(\hat{f})$ are not necessarily unbiased but they tend to have small biases compared to the magnitude of their standard error.
\end{itemize}

\begin{exampleblock}{Bias of $\hat{\sigma}^2$}
$$
\begin{array}{ll}
\hat{\sigma}^2&=\frac{1}{n} \sum_{i=1}^{n} (x_i-\overline{x})^{2}=\frac{1}{n} \sum_{i=1}^{n} ((x_i-\mu_f)+ (\mu_f-\overline{x}))^{2}\\
&\\
&=\left(\frac{1}{n}  \sum_{i=1}^{n} (x_i-\mu_f)^2 \right)-  (\overline{x}-\mu_f)^2\\
\end{array}
$$
The first term has an expected value of $\sigma_f^2$ and the second term has expected value $\sigma_f^2/n$. So the bias of  $\hat{\sigma}^2$ is:
$$
\mathrm{Bias}_{f}(\hat{\sigma}^2,\sigma_f^2)=\sigma_f^2-\frac{\sigma_f^2}{n}-\sigma_f^2=-\frac{\sigma_f^2}{n}
$$ 
\end{exampleblock}
}
